\documentclass[english]{article}
\newcommand{\G}{\overline{C_{2k-1}}}
\usepackage[latin9]{inputenc}
\usepackage{amsmath}
\usepackage{amssymb}
\usepackage{lmodern}
\usepackage{mathtools}
\usepackage[inline]{enumitem}
\usepackage{multicol}
%\usepackage{natbib}
%\bibliographystyle{plainnat}
%\setcitestyle{authoryear,open={(},close={)}}content...
\let\avec=\vec
\renewcommand\vec{\mathbf}
\renewcommand{\d}[1]{\ensuremath{\operatorname{d}\!{#1}}}
\newcommand{\pydx}[2]{\frac{\partial #1}{\partial #2}}
\newcommand{\dydx}[2]{\frac{\d #1}{\d #2}}
\newcommand{\ddx}[1]{\frac{\d{}}{\d{#1}}}
\newcommand{\hk}{\hat{K}}
\newcommand{\hl}{\hat{\lambda}}
\newcommand{\ol}{\overline{\lambda}}
\newcommand{\om}{\overline{\mu}}
\newcommand{\all}{\text{all }}
\newcommand{\valph}{\vec{\alpha}}
\newcommand{\vbet}{\vec{\beta}}
\newcommand{\vT}{\vec{T}}
\newcommand{\vN}{\vec{N}}
\newcommand{\vB}{\vec{B}}
\newcommand{\vX}{\vec{X}}
\newcommand{\vx}{\vec {x}}
\newcommand{\vn}{\vec{n}}
\newcommand{\vxs}{\vec {x}^*}
\newcommand{\vV}{\vec{V}}
\newcommand{\vTa}{\vec{T}_\alpha}
\newcommand{\vNa}{\vec{N}_\alpha}
\newcommand{\vBa}{\vec{B}_\alpha}
\newcommand{\vTb}{\vec{T}_\beta}
\newcommand{\vNb}{\vec{N}_\beta}
\newcommand{\vBb}{\vec{B}_\beta}
\newcommand{\bvT}{\bar{\vT}}
\newcommand{\ka}{\kappa_\alpha}
\newcommand{\ta}{\tau_\alpha}
\newcommand{\kb}{\kappa_\beta}
\newcommand{\tb}{\tau_\beta}
\newcommand{\hth}{\hat{\theta}}
\newcommand{\evat}[3]{\left. #1\right|_{#2}^{#3}}
\newcommand{\prompt}[1]{\begin{prompt*}
		#1
\end{prompt*}}
\newcommand{\vy}{\vec{y}}
\DeclareMathOperator{\sech}{sech}
\DeclarePairedDelimiter\abs{\lvert}{\rvert}%
\DeclarePairedDelimiter\norm{\lVert}{\rVert}%
%\newcommand{\dis}[1]{\begin{align*}
%	#1
%	\end{align*}}
\newcommand{\dis}[1]{$\displaystyle{#1}$}
\newcommand{\LL}{\mathcal{L}}
\newcommand{\RR}{\mathbb{R}}
\newcommand{\CC}{\mathbb{C}}
\newcommand{\NN}{\mathbb{N}}
\newcommand{\ZZ}{\mathbb{Z}}
\newcommand{\QQ}{\mathbb{Q}}
\newcommand{\Ss}{\mathcal{S}}
\newcommand{\BB}{\mathcal{B}}
\usepackage{graphicx}
\setlength{\columnseprule}{1pt}

% Swap the definition of \abs* and \norm*, so that \abs
% and \norm resizes the size \dis{\tan (\pi/6)}of the brackets, and the ake no more than 5 minutes on question 0}
%%\prob{0}\textbf{Oops! Looks like the American dollar went belly-up last week, and as a result, society has completely collapsed across the globe!\footnote{Maybe next time, consider using something less volatile as the world's reserve currency\textemdash for now though, tough luck.} Eventually, maybe civilization will rebuild, but for now, it's total chaos\textemdash there's no longer such thing as currency, half the city is on fire, and the police and army have both fractured into a complicated web of warring factions. You and your group have to band up to survive. In the space below, determine a) what your top priorities for survival should be in the immediate term and b) what role each member will play for your team.}
% starred version does not.
%\makeatletter
%\let\oldabs\abs
%\def\abs{\@ifstar{\oldabs}{\oldabs*}}
%content...
%\let\oldnorm\norm
%\def\norm{\@ifstar{\oldnorm}{\oldnorm*}}
%\makeatother
\newenvironment{subproof}[1][\proofname]{%
	\renewcommand{\qedsymbol}{$\blacksquare$}%
	\begin{proof}[#1]%
	}{%
	\end{proof}%
}

\usepackage{centernot}
\usepackage{dirtytalk}
\usepackage{calc}
%\newcommand{\prob}[1]{\setcounter{section}{#1-1}\section{}}
%\newcommand{\prt}[1]{\setcounter{subsection}{#1-1}\subsection{}}
%\newcommand{\pprt}[1]{{\textit{{#1}.)}}\newline}
%\renewcommand\thesubsection{\alph{subsection}}
\usepackage[]{titlesec}
%\titlelabel{\thetitle.)\quad}
\DeclarePairedDelimiter\floor{\lfloor}{\rfloor}
\makeatletter

%\newcommand*\pFqskip{8mu}
%\catcode`,\activecontent...
%\newcommand*\pFq{\begingroup
%	\catcode`\,\active
%	\def ,{\mskip\pFqskip\relax}%
%	\dopFq
%}
%\catcode`\,12
%\def\dopFq#1#2#3#4#5{%
%	{}_{#1}F_{#2}\biggl(\genf\dis{\tan (\pi/6)}rac..{0pt}{}{#3}{#4}|#5\biggr
%	)%
%	\endgroup
%}
\def\res{\mathop{Res}\limits}
% Symbols \wedge and \vee from mathabx
% \DeclareFontFamily{U}{matha}{\hyphenchar\font45}
% \DeclareFontShape{U}{matha}{m}{n}{content...
%       <5> <6> <7> <8> <9> <\dis{\tan (\pi/6)}10> gen * matha
%       <10.95> matha10 <12> <14.4> <17.28> <20.74> <24.88> matha12
%       }{}
% \DeclareSymbolFont{matha}{U}{matha}{m}{n}ake no more than 5 minutes on question 0}
%%\prob{0}\textbf{Oops! Looks like the American dollar went belly-up last week, and as a result, society has completely collapsed across the globe!\footnote{Maybe next time, consider using something less volatile as the world's reserve currency\textemdash for now though, tough luck.} Eventually, maybe civilization will rebuild, but for now, it's total chaos\textemdash there's no longer such thing as currency, half the city is on fire, and the police and army have both fractured into a complicated web of warring factions. You and your group have to band up to survive. In the space below, determine a) what your top priorities for survival should be in the immediate term and b) what role each member will play for your team.}
% \DeclareMathSymbol{\wedge}         {2}{matha}{"5E}
% \DeclareMathSymbol{\vee}           {2}{matha}{"5F}content...
% \makeatother

%\titlelabel{(\thesubsection)}
%\titlelabel{(\thesubsection)\quad}
\usepackage{listings}
\lstloadlanguages{[5.2]Mathematica}
\usepackage{babel}
\newcommand{\ffac}[2]{{(#1)}^{\underline{#2}}}
\usepackage{color}
\usepackage{amsthm}
\newtheorem{theorem}{Theorem}[section]
%\newtheorem*{theorem*}{Theorem}[section]
\newtheorem{conj}[theorem]{Conjecture}
\newtheorem{corollary}[theorem]{Corollary}
\newtheorem{example}[theorem]{Example}
\newtheorem{lemma}[theorem]{Lemma}
\newtheorem*{lemma*}{Lemma}


\newtheorem{proposition}[theorem]{Proposition}
\newtheorem*{prop*}{Proposition}
\newtheorem*{corollary*}{Corollary}


\newtheorem*{claim*}{Claim}
\newcommand{\claim}[1]{\begin{claim*} #1\end{claim*}}
%organizing theorem environments by style--by the way, should we really have definitions (and notations I guess) in proposition style? it makes SO much of our text italicized, which is weird.
\theoremstyle{remark}
\newtheorem{remark}{Remark}[theorem]
\newtheorem{caution}{Caution}[theorem]

\theoremstyle{definition}
\newtheorem{excs}{Exercise}
\newtheorem{prbm}[excs]{Problem}
\newtheorem{exle}[theorem]{Example}
\newtheorem{definition}[theorem]{Definition}
\newtheorem{notation}[theorem]{Notation}
\newtheorem*{notation*}{Notation}
\newtheorem*{goal}{Goal}
\newtheorem{fact}[theorem]{Fact}
%FINAL
\newcommand{\due}{6 June 2018} 
\RequirePackage{geometry}
\geometry{margin=.7in}
\usepackage{todonotes}
\title{Summer Calc 1.5 Notes}
\author{David DeMark}
\date{\due}
\usepackage{fancyhdr}
\pagestyle{fancy}
\fancyhf{}
\chead{Unit 1: Euler's Formula \& Limits I}
\rhead{\due}
\lhead{MATH 1271.5}
\cfoot{\thepage}
% %%
%%
%%
%DRAFT

%\usepackage[left=1cm,right=4.5cm,top=2cm,bottom=1.5cm,marginparwidth=4cm]{geometry}
%\usepackage{todonotes}inparaenum
% \title{MATH 8669 Homework 4-DRAFT}
% \usepackage{fancyhdr}
% \pagestyle{fancy}
% \fancyhf{}\dis{\tan (\pi/6)}
% \rhead{David DeMark}
% \lhead{MATH 8669-Homework 4-DRAFT}
% \cfoot{\thepage}

%PROBLEM SPEFICIC

\newcommand{\lint}{\underline{\int}}
\newcommand{\uint}{\overline{\int}}
\newcommand{\hfi}{\hat{f}^{-1}}
\newcommand{\tfi}{\tilde{f}^{-1}}
\newcommand{\tsi}{\tilde{f}^{-1}}
\newcommand{\PP}{\mathcal{P}}
\newcommand{\nin}{\centernot\in}
\newcommand{\seq}[1]{({#1}_n)_{n\geq 1}}
\usepackage{array}
\newcolumntype{M}[1]{>{\centering\arraybackslash}m{#1}}
\newcolumntype{N}{@{}m{0pt}@{}}
\input ArtNouvc.fd
\newcommand*\initfamily{\usefont{U}{ArtNouvc}{xl}{n}}
\newcommand{\ansl}{~\underline{\hspace{1.5cm}}}
\usepackage{pgfplots}
\pgfplotsset{my style/.append style={axis x line=middle, axis y line=
		middle, xlabel={$x$}, ylabel={$y$}, axis equal,width=\textwidth }}
\pgfplotsset{every x tick label/.append style={font=\small, yshift=0.5ex}}
\pgfplotsset{every y tick label/.append style={font=\small, xshift=0.5ex}}
%\usepackage{paralist}
\newcommand{\dlim}{\displaystyle\lim}
\begin{document}
\maketitle
\chapter{Unit 1: Euler's Formula \& Limits I}
\setcounter{section}{-1}
\section{Euler's Formula}
\begin{goal} Use Euler's Formula (Theorem \ref{thm:eufl}) to derive any number of trig identities.
	\end{goal}
\emph{Note that this isn't actually part of the standard pre-calc curriculum, but it is an immensely powerful tool which simplifies down quite a bit of pre-calc\textemdash frankly, it's almost offensive that this isn't taught during pre-calc. We may be able to prove Theorem \ref{thm:eufl} using Taylor series some time in August, but for now we'll need to just trust Euler on it.}

We'll be using the following fact over and over throughout this section:
\begin{fact}
	For any (real, complex, or otherwise!) $a,b,c$, we have that $a^b*a^c=a^{b+c}$.
\end{fact}

\begin{definition}
	Recall the imaginary number $i:=\sqrt{-1}$. 
\end{definition}

\begin{theorem}
\label{thm:eufl} For any real number $\theta$, $$e^{i\theta}=\cos(\theta)+i\sin(\theta)$$
\end{theorem}
\begin{caution}
	This is only true in radians! And is a compelling reason never to use degrees for much of anything\textellipsis
\end{caution}
\begin{fact}\label{fact:cplx=}
	We let $w=a+ib$ and $z=c+id$ be complex numbers (that is $w,z\in \CC$) with $a,b,c,d$ all real numbers (that is, $a,b,c,d\in \RR$). Then, $w=z$ if and only if $a=c$ and $b=d$.
\end{fact}

This is a somewhat obvious fact, but it will prove extremely useful\textemdash basically, if we can find two different ways to write the same complex number, we can then conclude that the real parts and the imaginary parts must be the same (because it's the same number!). We'll use this fact to show quite a few trig identities.

\begin{exle} Let $\theta$ be any real number.\label{ex1.0}
	Find formulas for $\sin(2\theta)$ and $\cos(2\theta)$ in terms of $\sin(\theta)$ and $\cos(\theta)$.
\end{exle}
\begin{proof}[Response]
We use Euler's Formula (Theorem \ref{thm:eufl}) to write $$e^{i(2\theta)}=\cos(2\theta)+i\sin (2\theta).$$
However, we also have by properties of exponentiation that $e^{i(2\theta)}=e^{2(i\theta)}=(e^{i\theta})^2$, and note that we already know how to expand out $e^{i\theta}$ (once again by Theorem \ref{thm:eufl}). That is, we can write
\begin{align*}e^{i(2\theta)}&=(e^{i\theta})^2\\
&=(\cos (\theta)+i\sin (\theta))^2\\
&=\cos^2(\theta) +2i\cos(\theta)\sin(\theta)+(i\sin (\theta))^2\\
&=(\cos^2\theta -\sin ^2\theta)+i(2\cos(\theta)\sin(\theta))
\end{align*}
Now, we've written the same complex number two different ways:
\begin{align*}e^{i(2\theta)}&=\cos(2\theta)+i\sin (2\theta)\\&=(\cos^2(\theta) -\sin ^2(\theta))+i(2\cos(\theta)\sin(\theta))\end{align*}
Thus, we can use Fact \ref{fact:cplx=} to now write that $\cos(2\theta)=cos^2(\theta)-\sin^2(\theta)$, and $\sin(2\theta)=2\cos (\theta)\sin(\theta).$
\end{proof}
\begin{excs}
Simplify the expression $e^{i\pi}+1$
\end{excs}
\begin{excs}
	Use the same techniques as in Example \ref{ex1.0} to find formulas for $\cos(3\theta)$ and $\sin(3\theta)$ in terms of $\cos(\theta)$ and $\sin(\theta)$. 
\end{excs}
\begin{excs} Let $\theta_1$ and $\theta_2$ be real numbers.
	Use the same techniques as in Example \ref{ex1.0} to find formulas for $\cos(\theta_1+\theta_2)$ and $\sin(\theta_1+\theta_2)$ in terms of $\sin(\theta_1)$, $\sin(\theta_2)$, $\cos(\theta_1)$, and $\cos(\theta_2)$. 
\end{excs}
\begin{excs}
	We know that in general, for any $x$, \begin{equation}e^x*e^{-x}=1.\label{eqrec}\end{equation}
	This is still true if $x$ is imaginary (or complex for that matter). Let $x=i\theta$. Reinterpret equation \eqref{eqrec} in terms of $\sin(\theta)$ and $\cos(\theta)$. What familiar trigonometric identity do we recover? (Hint: what symmetries do $\cos$ and $\sin$ have?)
\end{excs}
\section{Limits I: Formal Definition \& First Properties}
See: Stewart, \S2.4.
\begin{definition}\label{def:lim}
	Let $f$ be a function on the real numbers, and let $a$ and $L$ be real numbers. We say $\displaystyle\lim_{x\to a}f(x)=L$ if for any real number $\epsilon>0$, there exists some real number $\delta>0$ such that if $0<\abs{x-a}<\delta$, then $\abs{f(x)-L}<\epsilon$.
\end{definition}
\begin{remark}
Let's unpack that definition a bit. First, keep in mind that we should think of $\abs{x-a}$ as the \emph{distance} between $x$ and $a$. When we say \say{for any $\epsilon$, there exists a $\delta$,} that means that if \emph{we are given} some number $\epsilon$, then (theoretically) we can find some number $\delta$ with the property that if $x$ is \say{less than $\delta$ away from $a$,} then $f(x)$ is \say{less than $\epsilon$ away from $L$.} 

To illustrate this more concretely, let's add in some literal illustrations. Suppose we have the curve $y=f(x)$ as in figure \ref{fig:1graph1}, and suppose we are considering the statement $\displaystyle\lim_{x\to a} f(x)=L$.
\begin{figure}[h!]\centering
\includegraphics[scale=0.5,trim={0 4in 10mm 0},clip]{1graph1} \caption{The graph of $y=f(x)$ with $x=a$ and $y=L$ drawn in.\label{fig:1graph1}}
\end{figure}

Now, suppose we are given some real number $\epsilon>0$. $\epsilon$ corresponds to some region around $L$ on the $y$-axis (the region where $\abs{y-L}<\epsilon$). That region is drawn in red in figure \ref{fig:1graph2}

\begin{figure}[h!]\centering
	\includegraphics[scale=0.5,trim={0 4in 10mm 0},clip]{1graph2} \caption{The graph of $y=f(x)$ with the vertical region between $L-\epsilon$ and $L+\epsilon$ drawn in.\label{fig:1graph2}}
\end{figure}

\newpage Given such an $\epsilon$, we need to find some horizontal region around $x=a$ such that for any $x$ in that region, $f(x)$ is \say{close enough} to $L$. We can see from the graph that such a region indeed exists\textemdash note the part of the curve which sits between the two red lines. We choose some $\delta$ such that everything between $a-\delta$ and $a+\delta$ (that is, all $x$ with $\abs{x-a}<\delta$) has its corresponding point on the curve between $L-\epsilon$ and $L+\epsilon$. Such a $\delta$ and the corresponding region around $x=a$ is drawn in figure \ref{fig:1graph3}.

\begin{figure}[h!]\centering
	\includegraphics[scale=0.5,trim={0 4in 10mm 0},clip]{1graph3} \caption{The graph of $y=f(x)$, now with a horizontal region around $x=a$.\label{fig:1graph3}}
\end{figure}

Note that we could have chosen a different $\delta$\textemdash in particular, any smaller $\delta$ would work.
\end{remark}
\begin{exle}
	Use Definition \ref{def:lim} to show that $\displaystyle\lim_{x\to 2} 4x-3=5$.
\end{exle}
\begin{proof}[Response]
	 Let $x$ be arbitrary (that is, let it be any real number). Let's try writing out $f(x)-f(2)$.
	 \begin{align*}
	 	f(x)-f(2)&=(4x-3)-(4(2)-3)\\
	 	&=4x-3-4(2)+3\\
	 	&=4(x-2)
	 \end{align*}
%	 Thus, if we let $d=\abs{x-2}$, we have that $\abs{f(x)-f(2)}=\abs{4(x-2)}=4d$.
 Now, suppose we are given some $\epsilon>0$. We need to choose a $\delta$ such that if $\abs{x-2}<\delta$, then $\abs{f(x)-f(2)}<\epsilon$. But as we just saw, as long as we pick $0<\delta<\epsilon/4$, we get that if $\abs{x-2}<\delta$, then $\abs{f(x)-f(2)}=\abs{4(x-2)}=4\abs{x-2}<4\delta<4(\epsilon/4)=\epsilon$. So indeed, $\dlim_{x\to 2}(4x-3)=5$. 
\end{proof}
\end{document} 
